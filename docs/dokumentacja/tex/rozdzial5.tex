\newpage
\section{Wnioski}	%5
%Npisać wnioski końcowe z przeprowadzonego projektu, 
\nocite{www1}
\nocite{www2}
\nocite{www3}
\nocite{www4}
Realizacja projektu pozwoliła na stworzenie aplikacji umożliwiającej efektywne przetwarzanie i analizowanie danych energetycznych zapisanych w formacie CSV. Aplikacja spełnia założenia projektowe, zapewniając użytkownikom narzędzia do organizowania danych w hierarchiczną strukturę drzewa oraz przeprowadzania różnorodnych analiz. Podczas pracy nad projektem zidentyfikowano kluczowe wyzwania i rozwiązania, które miały istotny wpływ na jakość i funkcjonalność aplikacji.
\begin{enumerate}
    \item \textbf{Efektywność algorytmu i struktura danych} - Kluczowym elementem projektu było stworzenie wydajnego algorytmu organizowania danych w strukturę drzewa, co pozwala na szybki dostęp do informacji w różnych przedziałach czasowych. Zastosowanie struktury drzewa okazało się optymalnym rozwiązaniem, umożliwiającym łatwe zarządzanie danymi oraz przeprowadzanie operacji analitycznych. Dzięki temu użytkownicy mogą szybko uzyskać dostęp do wymaganych informacji oraz wykonywać obliczenia na danych w sposób efektywny.
    \item \textbf{Zarządzanie błędnymi danymi} - Projekt wymagał szczególnej uwagi w zakresie obsługi błędnych danych, takich jak puste linie, niekompletne rekordy czy powtarzające się wiersze. Zostały zaimplementowane mechanizmy walidacji, które umożliwiły odrzucanie błędnych danych i generowanie szczegółowych logów. Dzięki temu program jest odporny na błędy w danych wejściowych, co zapewnia jego niezawodność i stabilność w trakcie użytkowania.
    \item \textbf{Optymalizacja wydajności} - Przetwarzanie dużych zbiorów danych energetycznych wiąże się z koniecznością zoptymalizowania pamięci oraz algorytmów. W trakcie realizacji projektu zastosowano odpowiednie techniki optymalizacji, które pozwalają na efektywne przetwarzanie danych bez nadmiernego obciążania systemu. Zastosowanie języka C++ oraz narzędzi takich jak GitHub Copilot, pozwoliło na szybsze generowanie kodu oraz implementację algorytmów, co wpłynęło na skrócenie czasu rozwoju projektu.
\end{enumerate}
Projekt został zrealizowany zgodnie z założeniami i spełnia wymagania dotyczące efektywnego przetwarzania danych oraz analizy energii. Aplikacja zapewnia nie tylko odpowiednią funkcjonalność, ale również odporność na błędne dane wejściowe, wysoką wydajność oraz elastyczność w zarządzaniu danymi. Zastosowanie nowoczesnych narzędzi i technologii przyczyniło się do przyspieszenia procesu programowania oraz poprawy jakości aplikacji, co stanowi solidną podstawę do rozwoju projektu w przyszłości.
