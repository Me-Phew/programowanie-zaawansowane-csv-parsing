\newpage
\section{Projektowanie}		%3
%Napisać z jakich narzędzi będziemy korzystać (kompilator, język programowania), git, biblioteki dodatkowe, itp.
%Opisać szczegółowe ustawienia kompilatora (jeśli są), powiązania z bibliotekami, itp.
%Narysować graf, UML, diagram klas, schemat działania algorytmu
%Jeśli zadanie zakłada przedstawienie jakiegoś narzędzia (np. git, AI) należy opisać sposób jego używania
\subsection{Język programowania wykorzystany w projekcie}
W projekcie, zgodnie z wymaganiami, zostanie wykorzystany język C++. Język ten jest idealnym wyborem do implementacji złożonych aplikacji przetwarzających duże zbiory danych, takich jak hierarchiczne struktury drzewa przechowujące dane energetyczne. Dzięki możliwości manualnego zarządzania pamięcią, C++ pozwala na efektywne operowanie na danych o dużej objętości, co jest szczególnie istotne w przypadku analizy informacji zawartych w plikach CSV, które mogą obejmować tysiące lub miliony rekordów. C++ oferuje również zaawansowane wsparcie dla programowania obiektowego, co umożliwia stworzenie klas odpowiadających za węzły drzewa, operacje na danych oraz obsługę plików wejściowych i wyjściowych. Dzięki temu kod będzie modularny, przejrzysty i łatwiejszy do rozbudowy w przyszłości.
\subsection{Narzędzia i technologie wykorzystane w projekcie}
W projekcie wykorzystano następujące narzędzia i technologie, które wspierają zarówno rozwój aplikacji, jak i zapewniają jej odpowiednią jakość oraz funkcjonalność:
\begin{itemize}
    \item \textbf{Google Test} - Framework do testowania jednostkowego, który umożliwia automatyczne testowanie poprawności działania aplikacji w różnych scenariuszach. Google Test pozwala na łatwą integrację z systemami Continuous Integration (CI), co zapewnia automatyczne uruchamianie testów na każdym etapie rozwoju projektu. W projekcie służy do weryfikacji poprawności przetwarzania danych oraz testowania funkcji operujących na strukturze drzewa.
    \item \textbf{CMake} - Narzędzie do konfiguracji procesu budowania projektu. CMake generuje odpowiednie pliki konfiguracyjne dla różnych systemów operacyjnych i kompilatorów, co zapewnia elastyczność i łatwość w budowaniu aplikacji na różnych platformach, takich jak Windows i Linux. Dzięki CMake możliwe jest utrzymanie jednej, spójnej konfiguracji procesu budowy, co upraszcza zarządzanie projektem.
    \item \textbf{Ninja} - Lekki system budowania, który współpracuje z CMake i zapewnia szybkie kompilacje. Ninja został wybrany ze względu na mniejsze zużycie zasobów systemowych w porównaniu do tradycyjnych systemów budowania, co znacząco przyspiesza proces kompilacji i umożliwia efektywne zarządzanie czasem w trakcie rozwijania projektu.
    \item \textbf{Doxygen} -  Narzędzie do generowania dokumentacji na podstawie komentarzy w kodzie źródłowym. Doxygen automatycznie tworzy dokumentację w formacie HTML, co pozwala na łatwiejsze zrozumienie struktury i działania kodu. Dokumentacja generowana przez Doxygen jest dostępna na stronie projektu, co umożliwia łatwiejsze przeglądanie i korzystanie z niej przez przyszłych programistów oraz użytkowników.
    \item \textbf{GitHub Actions} - Platforma CI/CD, która umożliwia automatyczne uruchamianie testów, generowanie dokumentacji oraz wdrażanie aplikacji na repozytorium GitHub. GitHub Actions w tym projekcie służy do uruchamiania testów jednostkowych, weryfikacji konwencji commitów, generowania dokumentacji oraz publikacji wersji na GitHub Pages. Dzięki tej platformie cały proces rozwoju aplikacji jest zautomatyzowany, co znacząco zwiększa efektywność i zapewnia jakość kodu.
    \item \textbf{Github Copilot} - to narzędzie oparte na sztucznej inteligencji, które zostało stworzone przez GitHub we współpracy z OpenAI. Jego głównym celem jest wspomaganie programistów w codziennej pracy poprzez automatyczne sugerowanie fragmentów kodu, funkcji czy całych bloków logicznych na podstawie kontekstu i opisów wprowadzonych przez użytkownika. Copilot działa jako rozszerzenie do popularnych edytorów kodu, takich jak Visual Studio Code, i jest w stanie uczyć się na podstawie istniejącego kodu, co pozwala na dostosowywanie sugestii do bieżącego projektu.
\end{itemize}
\subsection{Zarządzanie projektem i współpraca w zespole}
W projekcie szczególną uwagę poświęcono efektywnemu zarządzaniu procesem programowania oraz współpracy w zespole. Dzięki zastosowaniu systemu kontroli wersji Git oraz platformy GitHub, udało się zapewnić pełną śledzenie zmian w kodzie oraz umożliwić współpracę wielu programistów nad różnymi elementami aplikacji w sposób zorganizowany i bezpieczny. 
