\newpage
\section{Ogólne określenie wymagań}		%1
%Określenie celu pracy, co chcemy uzyskać, jakie przewidujemy wyniki

Celem projektu jest implementacja i przetestowanie programu w języku C++ umożliwiającego analizę danych zawartych w pliku CSV. Projekt obejmuje stworzenie modułowego systemu, który umożliwia wczytywanie danych do pamięci komputera, ich przetwarzanie oraz analizę według określonych kryteriów. Głównym założeniem jest zaimplementowanie mechanizmów pozwalających na grupowanie danych w strukturze drzewa, gdzie poszczególne węzły odpowiadają hierarchii czasowej (rok, miesiąc, dzień, ćwiartka doby).
\subsection{Cel pracy}
Głównym celem projektu jest:
\begin{itemize}
    \item Implementacja programu do analizy danych z pliku CSV w języku C++ z wykorzystaniem struktury modularnej, gdzie każda klasa i funkcjonalność są umieszczone w oddzielnych plikach źródłowych.
    \item Zastosowanie struktury drzewa hierarchicznego do organizacji danych według jednostek czasowych (rok, miesiąc, dzień, ćwiartka doby) oraz zaimplementowanie wzorca projektowego iterator do poruszania się po drzewie.
    \item Opracowanie mechanizmów obliczeniowych umożliwiających użytkownikowi wykonywanie operacji takich jak obliczanie sum, średnich, porównań oraz wyszukiwanie danych w określonych przedziałach czasowych.
    \item Zabezpieczenie programu przed błędami w plikach CSV poprzez bieżącą analizę danych, odrzucanie niepoprawnych rekordów i tworzenie logów z procesu wczytywania.
    \item Przeprowadzenie testów jednostkowych za pomocą frameworka GoogleTest w celu zapewnienia poprawności działania programu w różnych scenariuszach.
    \item Opracowanie szczegółowej dokumentacji technicznej projektu przy użyciu narzędzi LaTeX i Doxygen oraz publikacja kodu źródłowego w serwisie GitHub.
    \item Stworzenie interaktywnego interfejsu użytkownika, który pozwala na wybór operacji z poziomu menu, w tym wczytywanie danych, analizę oraz ich zapis i odczyt w formacie binarnym.
\end{itemize}
\newpage
\subsection{Zakładane wyniki}
Oczekiwane efekty programu to:
\begin{itemize}
    \item Poprawna i wydajna implementacja programu do analizy danych z pliku CSV, który umożliwia wczytywanie, organizowanie i przetwarzanie danych w hierarchicznej strukturze drzewa z uwzględnieniem jednostek czasowych.
    \item Mechanizmy obliczeniowe pozwalające na wykonywanie operacji takich jak obliczanie sum, średnich, porównań oraz wyszukiwanie rekordów zgodnych z podanymi kryteriami, w tym tolerancji błędu i zakresu czasowego.
    \item Zbiór testów jednostkowych pokrywających kluczowe funkcjonalności programu, w tym obsługę poprawnych i niepoprawnych danych, dokładność wyników analiz oraz zachowanie programu w nietypowych scenariuszach, takich jak brak ciągłości czasowej w danych.
    \item Interaktywne menu użytkownika, które pozwala na wybór i realizację operacji, takich jak analiza danych, zapis i odczyt w formacie binarnym oraz wczytywanie danych z plików CSV.
    \item Szczegółowa dokumentacja techniczna i użytkownika opracowana w narzędziach LaTeX i Doxygen, obejmująca opis struktury programu, założeń projektowych i instrukcji obsługi.
    \item Repozytorium projektu w serwisie GitHub, zawierające kod źródłowy, testy jednostkowe oraz konfigurację do automatyzacji budowania i testowania za pomocą CMake i GoogleTest.
\end{itemize}
